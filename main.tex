% !TEX program = xelatex
\documentclass{ctexart}
\usepackage{xcolor} % 在latex中使用颜色
\usepackage{booktabs,tabularx,multicol} % 制作表格
\usepackage{framed} % 制作文本框
\usepackage{amsmath,amsthm,amssymb,amsfonts}    % 数学符号与字体

\usepackage[colorlinks,
            linkcolor=blue       %%修改此处为你想要的颜色
            %anchorcolor=blue,  %%修改此处为你想要的颜色
            %citecolor=blue,        %%修改此处为你想要的颜色,例如修改blue为red
            ]{hyperref}
\usepackage[left=2.0cm, right=2.0cm, top=2.5cm, bottom=2.5cm]{geometry} % 调整页边距
\usepackage{appendix}   % 附录环境
\usepackage{subfig,graphicx}    % 插入照片
\usepackage{float}      % 固定浮动体
\usepackage{lipsum,zhlipsum} %生成一些测试文本
%---------------优雅的插入MATLAB代码---------%
\usepackage{listings,matlab-prettifier} % MATLAB 美化包
\lstset{
        style=Matlab-editor,
        basicstyle=\mlttfamily,
        escapechar=`,
        numbers      = left,
        numbersep    = 5pt,
        numberstyle  = \small\color{red},
        frame        = single,
        keepspaces   = true,
        tabsize      = 4,
        breaklines = true,
}
%-------------标题-------------%
\title{数学小结论}
\date{\today}
\author{张阳}

%-----------做一些设置-----------%
\numberwithin{equation}{section}    % 公式标号与section的编号挂钩
\punctstyle{kaiming}    % 调整标点符号大小

%------------自定义一些命令------------%
\newcommand{\upcite}[1]{\textsuperscript{\textsuperscript{\cite{#1}}}}
\newcommand*{\dif}{\mathop{}\!\mathrm{d}}
\def\degree{${}^{\circ}$}

%---------配置环境------------%
\newtheorem{definition}{定义}
\newtheorem{theorem}{定理}

% \newenvironment{solution}{\par\noindent\textbf{解答. }}{\par}
% \newenvironment{note}{\par\noindent\textbf{题目\arabic{problemname}的注记. }}{\par}

%-------------可控列宽的表格--------%
\newcolumntype{R}[1]{>{\raggedright\arraybackslash}p{#1}}
\newcolumntype{C}[1]{>{\centering\arraybackslash}p{#1}}
\newcolumntype{L}[1]{>{\raggedleft\arraybackslash}p{#1}}

%------------- 使得矩阵的最大大小超过 10 x 10-----------------%
\addtocounter{MaxMatrixCols}{10}
\begin{document}
\maketitle
\tableofcontents
\newpage
\section{定义}
\begin{definition}[对角占优矩阵]
    对角占优矩阵是指一矩阵的每一横行,对角线上元素的大小大于或等于同一横行其他元素大小的和,一矩阵A为对角占优矩阵若

\begin{equation}
|a_{ii}|\geq \sum _{j\neq i}|a_{ij}|\quad {\text{for all }}i
\end{equation}
其中$a_{ij}$为第 $i$ 行第 $j$ 列的元素。

上述的定义中用到大于等于,其条件较松,因此有时会称为弱对角占优矩阵,若上述的定义用大于代替大于等于,则称为强对角占优矩阵。对角优势矩阵可以指弱对角占优矩阵,也可以指强对角占优矩阵,视上下文而定。
\end{definition}

\begin{definition}[范数]
    
\end{definition}
\begin{definition}[距离]
    设 $\mathbb{X}$ 为一非空集合,如果对于 $\mathbb{X}$ 中任给的两个元素 $x,y$,均有一个确定的实数,记为 $\rho(x,y)$,与它们对应且满足下面三个条件:
    \begin{itemize}
        \item 非负性:$\rho(x,y)\geqslant0,\rho(x,y)=0$ 的充分必要条件是 $x=y$。
        \item 对称性:$\rho(x,y)=\rho(y,x)$。
        \item 三角不等式:$\rho(x,y)\leqslant\rho(x,z)+\rho(z,y)$,这里 $z$ 也是 $\mathbb{X}$ 中的任意一个元素
    \end{itemize}
    则称 $\rho$ 是 $\mathbb{X}$ 上的一个距离,而称 $\mathbb{X}$ 是以 $\rho$ 为距离的距离空间,记为 $(X,\rho)$
\end{definition}

\begin{definition}[内积]
    
\end{definition}
\section{等式}

\begin{theorem}{Stirling公式}
    
\end{theorem}
\begin{theorem}{Wallis公式}
    
\end{theorem}
\begin{theorem}{和差化积公式}
    
\end{theorem}
\begin{theorem}{积化和差公式}
    
\end{theorem}
\begin{theorem}
    \begin{equation}
        \label{eqa:sin积分公式}
        \begin{aligned}
        \int_0^{\pi/2}\sin^n\varphi d\varphi=\left\{ \begin{aligned}
            \frac{(2m-1)!!}{(2m)!!}\cdot\frac{\pi}{2},\qquad&n=2m\\
            \frac{(2m)!!}{(2m+1)!!},\qquad\phantom{\cdot\frac{\pi}{2}}&n=2m+1\end{aligned}\right.
        \end{aligned}
    \end{equation}
\begin{proof}
\begin{enumerate}
\item 对于 $n=1$,$\int_0^{\pi/2}\sin \varphi\dif \varphi=1$,结论显然成立

\item 对于 $n=2$:

\begin{equation}
    \begin{aligned}
        \int_0^{\pi/2}\sin^2\varphi\dif\varphi&=\int_0^{\pi/2}\frac{1-\cos2\varphi}{2}\dif\varphi\\
        &=\left.\frac{x-\frac{1}{2}\sin2\varphi}{2}\right|_0^{\pi/2}\\
        &=\frac{\pi}{4}
    \end{aligned}
\end{equation}
结论也成立

\item 对于 $n>2$ 的情况:
\begin{equation}
    \begin{aligned}
        I = \int_0^{\pi/2}\sin^n\varphi\dif\varphi&=-\int_0^{\pi/2}\sin^{n-1}\varphi\dif\cos\varphi\\
        &=\left.-\sin^{n-1}\varphi\cos\varphi\right|_0^{\pi/2}+(n-1)\int_0^{\pi/2}\cos^2\varphi\sin^{n-2}\varphi\dif\varphi\\
        &=(n-1)\int_0^{\pi/2}\sin^{n-2}\varphi\dif\varphi-(n-1)I
    \end{aligned}
\end{equation}
所以
\begin{equation}
    \label{eqa:sin积分公式推导}
    I=\frac{n-1}{n}\int_0^{\pi/2}\sin^{n-2}\varphi\dif\varphi
\end{equation}
利用 \eqref{eqa:sin积分公式推导} 不难计算验证原结论。
\end{enumerate}

综上 \eqref{eqa:sin积分公式} 对任意 $n\in\mathbb{N}$ 都成立。
\end{proof}
\end{theorem}

\begin{theorem}{二阶常系数齐次线性递推数列通项公式}
https://zhuanlan.zhihu.com/p/33854447
\end{theorem}

\section{不等式}
\begin{theorem}{闵可夫斯基不等式}
    5
\end{theorem}
\begin{theorem}{柯西不等式}
\begin{equation}
    \left(\sum_{k=1}^na_kb_k\right)^2\leqslant\sum_{k=1}^na_k^2\cdot\sum_{k=1}^nb_k^2
\end{equation}
其中 $a_k,b_k(k=1,2,\cdots,n)$ 均为实数。
\end{theorem}
\begin{proof}
    任取实数 $\lambda$,则
\begin{equation}
    0\leqslant\sum_{k=1}^n(a_k+\lambda b_k)^2=\sum_{k=1}^na_k^2+2\lambda\sum_{k=1}^na_kb_k+\lambda^2\sum_{k=1}^nb_k^2
\end{equation}
右端是 $\lambda$ 的二次三项式。上述不等式表明,它对于 $\lambda$ 的一切实数值都是非负的,故其判别式不大于零,即

\begin{equation}
    \left(\sum_{k=1}^na_kb_k\right)^2\leqslant\sum_{k=1}^na_k^2\cdot\sum_{k=1}^nb_k^2
\end{equation}
\end{proof}
\begin{theorem}{琴生不等式}
    1
\end{theorem}
\begin{theorem}{柯西–施瓦茨不等式}
    
\end{theorem}
\begin{theorem}{伯努利不等式}
   \begin{equation}
    (1+x)^{a}> 1+ax
   \end{equation}
其中,$a$ 为正实数,$x\geqslant-1$
\end{theorem}
\begin{proof}
    设 $f(x)=(1+x)^a-ax-1$,于是 $f'(x)=a(1+x)^{a-1}-a$,$f''(x)=a(a-1)(1+x)^{a-2}>0$, 而 $f'(0)=0$,所以 $f(x)\geqslant f(0)=0$。
\end{proof}
\begin{theorem}
    设 $a,b\geqslant 0 $,$p,q$ 为满足 $\frac{1}{p}+\frac{1}{q}=1$ 的正数,则
    \begin{equation}
        ab\leqslant\frac{1}{p}a^p+\frac{1}{q}b^q
    \end{equation}
\end{theorem}
\begin{proof}
    考虑函数 $f(x)=\ln x$ ,由 $f''(x)=-\frac{1}{x^2}<0$ 所以 $f(x)$ 在 $(0,+\infty)$ 是严格上凸函数。由上凸函数的性质可得:
    
    \begin{equation}
    \ln(ab)=\frac{1}{p}\ln(a^p)+\frac{1}{q}\ln(b^q)\leqslant\ln(\frac{1}{p}a^p+\frac{1}{q}a^q)
    \end{equation}

    由 $\ln(x)$ 的单调性即可得到原不等式。
    
\end{proof}
\section{代数}
\begin{theorem}
    若 $A$ 为实方阵,则 $A$ 为实对称阵当且仅当 $AA^T=A^2$
\end{theorem}
\begin{proof}
    \href{https://www.bilibili.com/video/BV1Ev411174v}{一个有关迹的例题在证明中的运用}
\end{proof}
\begin{theorem}
    $A$ 为实对称矩阵,则
\begin{equation}
    A=0 \iff {\rm tr}(AA^T)=0
\end{equation}
\end{theorem}
\begin{theorem}
    设 $n$ 阶矩阵 $A$ 满足

    \begin{equation}
        a_{ii}>\sum_{j\neq i}\lvert a_{ij}\rvert
        \label{eqa:主对角占优}
    \end{equation}
    则 $\lvert a_{ij}\rvert>0$ 
\end{theorem}
\begin{proof}
先证明 $\lvert A\rvert\neq 0$
    \href{https://www.bilibili.com/video/BV1JY411h71h}{华中科大2022高代:主对角占优矩阵}

若 $\lvert A\rvert=0$,则存在一个非零向量 $x$,满足:

\begin{equation}
    Ax=0
    \label{eqa:Ax=0}
\end{equation}
记 $x_m=\max_{i}x_i$,于是,考虑线性方程组 \eqref{eqa:Ax=0} 中的第 $m$ 个方程,有:

\begin{equation}
a_{m1}x_1+a_{m2}x_2+\cdots+a_{mn}x_n=0
\end{equation}

将除第 $m$ 项外的其他项移到等式右边,再将等式两边除以 $m$,得到:

\begin{equation}
    x_m = -\frac{a_{m1}}{a_{mm}} x_1-\frac{a_{m2}}{a_{mm}} x_2-\cdots- \frac{a_{m(m-1)}}{a_{mm}}x_{m-1}-\frac{a_{m(m+1)}}{a_{mm}}x_{m+1}-\cdots-\frac{a_{mn}}{a_{mm}}x_n
    \end{equation}

    于是

\begin{equation}
    \begin{aligned}
        \lvert a_{mm} \rvert &=\left| \frac{x_1}{x_m} a_{m1}+\frac{x_2}{x_m}a_{m2}+\cdots+ \frac{x_{m-1}}{x_m}a_{m(m-1)}+\frac{x_{m+1}}{x_m}a_{m(m+1)}+\cdots+\frac{x_{n}}{x_m}a_{mn}\right|\\
        &<\left|\frac{x_1}{x_m} a_{m1}\right|+\left|\frac{x_2}{x_m} a_{m2}\right|+\cdots+ \left|\frac{x_{m-1}}{x_m}a_{m(m-1)}\right|+\left|\frac{x_{m+1}}{x_m}a_{m(m+1)}\right|+\cdots+\left|\frac{x_n}{x_m}a_{mn}\right|\\
        &\leqslant |a_{m1}|+|a_{m2}|+\cdots+|a_{m(m-1)}|+|a_{m(m+1)}|+\cdots+|a_{mn}|\\
        &=|\sum_{i\neq m}{a_{mi}}|
    \end{aligned}
    \end{equation}

这与公式 \eqref{eqa:主对角占优} 矛盾。于是假设不成立。即 $|A|\neq 0$
\end{proof}
\end{document}